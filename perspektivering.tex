\section{Perspektivering}
Under flere af undersøgelserne havde det været fordelagtigt at udføre forsøget flere gange for at få en bedre ide om konfidensintervallet for vores data. Dette er specielt sandt for MIC--målingen, da det villet øge reproducerbarheden af undersøgelsen drastisk, samt præcissere den indsamlede data til værdier der potentielt lagde nærmere dem fundet teoretisk.

Udførslen af flere agar--diffusionsforsøg kunne også have yderligere underbygget vores data, da der kunnet have været mere eksperimentering med parabenkoncentrationerne i papirdiskene, hvilket kunnet have bidraget til opstillingen af bedre estimater for MIC--værdier, samt undersøgelse af den mindste effektive koncentration, da der generelt under MIC--forsøget observeredes en rimeligt drastisk stigning fra punktet svarende til MIC100 til den næste absorbansmåling, hvilket selvfølgelig giver mening da koncentrationen halveres fra brønd til brønd.

Undersøgelse af MIC--værdier for parabenblandinger, samt blandinger bestående af parabener og andre konserveringsmidler kunnet også have været interessant, da disse ofte benyttes i industrien. Derudover kunnet parabenernes effektivitet også visualiseres og kvantificeres i en større kontekst end blot deres egen molekylgruppe ved samligning med andre stoffer. Specielt undersøgelse med henblik på samligning af ``naturlige'' konserveringsmidler (salt, citronsyre, honning, osv.) og syntetiske konserveringsmidler kunnet være interessant for at få en ide om hvilke fordele (og ulemper?) de syntetiske produkter har ift.\ de naturlige.

Dette kunnet dog have nogle begrænsninger, specielt ift.\ undersøgelsen af konserveringsmidlerne betydning for fødevarernes fysiske karakteristika og smag, da parabener er kategoriserede som værende potentielt sundhedsskadelige, hvilket villet gøre kvalitativ samligning af fødevarer præserveret med naturlige vs.\ syntetiske konserveringsmidler svært.
