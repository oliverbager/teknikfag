\section{Konklusion}
Vores smeltepunkter for både det kommercielle-- og produktet fremstilt ved syntese afveg meget lidt fra hinanden, hvilket er indikativt for renhed grundet manglen på potentielle biprodukter. Derudover afveg vores retentionsfaktor for udførte TLC--analyser også meget lidt fra det for kommercielle produkt. 

Samtidigt viste de mere selektive spektroskopiske analyser også minimale afvigelser fra teoretiske-- og praktiske spektra, hvilket bekræfter vores mere kvantitative analysers i at det dannede produkt er det vi ønskede.

De hæmmende egenskaber af parabenerne var også som forventede. De bestemte MIC--værdier afveg relativt meget ved \textit{bacillus cereus}--mediet, dog passede de nærmest en--til--en med \textit{escherichia coli}--mediet. Konsekvent hæmning observeredes dog for begge parabener, hvilket er hvad der forventes.

Med udgangspunkt i undersøgelsen af de fysiske karakteristika--, spektroskopiske molekylinteraktioner samt den hæmmende effekt er det derfor sikkert at konkludere at det dannede produkt er hvad vi ønskede at danne ved syntesen, samt stor renhed. Syntesen samt den tilhørende analyse af produktet dannet derved har derfor været en succes. Samtidigt er det lykkedes os at eftervise parabenernes hæmmende effekt på mikroorganisme vækst.
