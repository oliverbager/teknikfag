\section{Indledning}
    Konserveringen af fødevarer har gennem tiden spillet en enorm rolle for menneskets overlevelse som art. At kunnet bevare fødevarer gennem perioder med mangel, har gjort det muligt at have adgang til mad hele året rundt. 

    I nutiden, hvor forsyningen af mad ikke er et problem grundet globaliseringen, har bevaringen af mad fået en ny rolle, nemlig at muliggøre transportation og langvarig opbevaring af egnede fødevarer \parencite{Elis2019}.

    Før i tiden benyttedes oftest ``naturlige'' metoder til at konservere fødevarer, eksempler på disse kunne være: 
    \begin{itemize}
        \item[-] saltning af kød, hvilket medfører dehydrering og derved værre miljø for bakterievækst
        \item[-] tørring af frugt, grøntsager, medfører igen dehydrering
        \item[-] syltning, placerer fødevaren i et syrligt miljø hvilket minimerer bakterievækst
        \item[-] frysning, reducerer enzymaktiviteten drastisk og fryser vandet hvilket gør det ubrugeligt for mikroorganismer
        \item[-] rygning, dræber overfladebakterier og tørrer den ud, røg indeholder formaldehyd og alkohol som har bevarende virkning
    \end{itemize}
    I dag benyttes ofte i stedet syntetiske konserveringsmidler der produceres i laboratorier, grundet deres bedre effektivtet, stabilitet og omkostning. Derudover minimeres også varians mellem partier af konserveringsmidler, hvilket medfører konsekvent bevaring.

    Dæmonisering af syntetiske tilsætningsstoffer har dog medført at mange er bange for de syntetiske konserveringsmidlers indvirkning på menneskets sundhed, på trods af manglende empirisk data der understøtter dette. 

    Dette har medført at nogle producenter er begyndt at bevæge sig mod de ``naturlige''-- eller alternative konserveringsmidler med mindre data der understøtter deres inerte helbredspåvirkning for at appelere til forbrugerne, hvilket kan vise sig at have konsekvenser senere hen grundet den større eksponering til mindre velundersøgte-- og potentielt sundhedsskadelige stoffer.

