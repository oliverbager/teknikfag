\section{Indledning}
    Konserveringen af fødevarer har gennem tiden spillet en enorm rolle for menneskets overlevelse som art. At kunnet bevare fødevarer gennem perioder med mangel, har gjort det muligt at have adgang til mad hele året rundt. 

    I nutiden, hvor forsyningen af mad ikke er et problem grundet globaliseringen, har præserveringen af mad fået en ny rolle: at muliggøre transportation og langvarig opbevaring af egnede fødevarer \parencite{Elis2019}.

    Før i tiden benyttedes oftest ``naturlige'' metoder til at konservere fødevarer, eksempler på disse kunne være: 
    \begin{itemize}
        \item[-] saltning af kød, hvilket medfører dehydrering og derved værre miljø for bakterievækst
        \item[-] tørring af frugt, grøntsager, medfører igen dehydrering
        \item[-] syltning, placerer fødevaren i et syrligt miljø hvilket minimerer bakterievækst
        \item[-] frysning, reducerer enzymaktiviteten drastisk og fryser vandet hvilket gør det ubrugeligt for mikroorganismer
        \item[-] rygning, dræber overfladebakterier og tørrer den ud, røg indeholder formaldehyd og alkohol som har bevarende virkning
    \end{itemize}
    Alle ovenstående ændrer på både fødevarernes smagsprofil-- samt deres fysiske struktur og derved konsistens. Derudover egner de sig \textit{ikke} til vandige fødevarer, hvilket gør dem enormt svære at præservere ved sådanne metoder. 

    I dag benyttes ofte i stedet syntetiske konserveringsmidler der produceres i laboratorier, grundet deres bedre effektivtet, stabilitet og omkostning. En anden betydelig fordel ved disse er evnen til at præservere fødevarer uden at ændre på deres fysiske karakteristika eller smagsprofil, hvilket gør præservering af fødevarer ligegyldigt form muligt.

    Dæmonisering af syntetiske tilsætningsstoffer har dog medført at mange er bange for de syntetiske konserveringsmidlers indvirkning på menneskets sundhed, på trods af manglende empirisk data der understøtter dette, hvilket har medført at producenter er begyndt at benytte mere ``naturlige'' (eller mindre omtalte), konserveringsmidler, hvilket på sigt kan have konsekvenser grundet den større eksponering til mindre velundersøgte stoffer.

